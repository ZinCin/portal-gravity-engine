\documentclass{article}
\usepackage{graphicx} % Required for inserting images
\usepackage[english,russian]{babel} % Connect russian

\title{Физика и методы расчетов}
\author{ZinCin }
\date{February 2026}

\begin{document}

\maketitle

\section{Постановка задачи}

В двумерной области $\Omega = [0, W] \times [0, H]$ найти потенциал каждоей точки $\varphi_{x,y}$. Для воссоздания потенциала, схожего с земным, зададим следующие граничные условия:

1. Верхняя граница ($y=H$): фиксированный потенциал 1.0
2. Нижняя граница ($y=0$): фиксированный потенциал 0.0
3-4. Боковые границы ($x=0$ и $x=W$): Условие Неймана $\frac{\delta\varphi}{\delta n}=0 \Rightarrow \frac{\delta\varphi}{\delta x}=0$

\section{Уравнение Лапласа}

В электростатике потенциал $\varphi$ в областях, свободных от зарядов, удовлетворяет уравнению Лапласа:

$$\nabla^2\varphi = \frac{\delta^2\varphi}{\delta x^2}+\frac{\delta^2\varphi}{\delta y^2} = 0$$

Введём равномерную сетку с шагом $h=1$ между узлами. Узлы сетки имеют координаты $0, 1, 2,...,W-1,W$ и $0, 1, 2,...,H-1,H$

$$\left.\frac{\partial^2\varphi}{\partial x^2}\right|_{i,j} \approx \varphi_{i,j-1}-2\varphi{i,j}+\varphi_{i,j+1},$$

$$\left.\frac{\partial^2\varphi}{\partial y ^2}\right|_{i,j} \approx \varphi_{i-1,j}-2\varphi{i,j}+\varphi_{i+1,j},$$

Подставляем в уравнение Лапласа:

$$(\varphi_{i-1,j}-2\varphi_{i,j}+\varphi_{i+1,j}) + (\varphi_{i,j-1}-2\varphi_{i,j}+\varphi_{i,j+1}) = 0,$$

получая пятиточечный шаблон:

$$\varphi{i-1,j}+\varphi_{i+1,j}+\varphi_{i,j-1}+\varphi_{i,j+1}-4\varphi_{i,j}=0,$$

или что то же самое:

$$\boxed{\varphi_{i,j}=\frac{1}{4}(\varphi_{i-1,j} + \varphi_{i+1,j} + \varphi_{i,j-1} + \varphi_{i,j+1})}$$

Откуда следует, что в каждой внутренней точке значение потенциала должно равняться среднему арифметическому четырёх соседей.

\section{Итерационные методы решения}

В отличии от решения Optozorax, я не использовал Метод Конечных Элементов (МКЭ), который помогает дискретизировать пространство и создаёт численный метод решения дифференциальных уравнений, а использовал другие итерационные методы.
Далее будут рассмотрены методы вычисления потецниала во всей симуляции, а верхними индексами k и k+1 будут обозначаться потенциал в текущем и новом кадрах соответственно.

\subsection{Метод Якоби}

$$\varphi_{i,j}^{k+1} = \frac{1}{4}(\varphi_{i-1,j}^{k} + \varphi_{i+1,j}^{k} + \varphi_{i,j-1}^{k} + \varphi_{i,j+1}^{k}),$$

Это приемлемый метод, который я применял раньше, однако сейчас я нашел более эффективные методы, описанные ниже.

\subsection{Метод Гаусса–Зейделя}

На каждой итерации мы проходим по всем внутренним узлам и обновляем значение по формуле:

$$\varphi_{i,j}^{k+1} = \frac{1}{4}(\varphi_{i-1,j}^{k+1} + \varphi_{i+1,j}^{k} + \varphi_{i,j-1}^{k+1} + \varphi_{i,j+1}^{k}),$$

где верхний индекс k означает итерацию. При этом используются уже посчитанные на текущей итерации значения слева и сверху (если обход идёт построчно сверху вниз, слева направо), что позволяет нам использовать метод Red-Black SOR.

\subsection{Метод последовательной верхней релаксации SOR}

SOR ускоряет сходимость, вводя параметр релаксации $\omega$,удовлетворяющий неравенству $1 < \omega < 2$.
Сперва вычисляется значение по Гауссу-Зейделю $\tilde\varphi$, а затем новое значение как взвешенная сумма старого потнциала $\varphi$ и нового $\tilde\varphi$:

$$\tilde\varphi_{i,j} = \frac{1}{4}(\varphi_{i-1,j}^{k+1} + \varphi_{i+1,j}^{k} + \varphi_{i,j-1}^{k+1} + \varphi_{i,j+1}^{k}),$$

$$\varphi_{i,j}^{k+1} = \varphi_{i,j}^k(1 - \omega) + \tilde\varphi_{i,j}\omega,$$

или эквивалентная форма:

$$\varphi_{i,j}^{k+1} = \varphi_{i,j}^k + \omega(\tilde\varphi_{i,j}-\varphi_{i,j}^k),$$

При $\omega = 1$ метод в точности совпадает с Гауссом-Зейделем (т.к. $(1 - \omega)=(1-1)=0$), а при $1 < \omega < 2$ потенциал перенаправляется в сторону среднего, что ускоряет сходимость задач с превосходством низкочастотных компонент ошибки (с точностью до погрешности).

\subsection{Red-Black SOR}

В моем проекте я использую Red-Black SOR. Чтобы избежать зависимости порядка обхода и обеспечить векторизацию, я применяю шахматное упорядочивание. Разделим все внутренние узлы на два множества: красные (сумма индексов чётна), и чёрные (сумма индексов нечётна). Важно, что все соседи красных узлов - черные, и наоборот.

Итерация проходит в два полушага:

1. Обновляются все красные узлы, используя старые значения чёрных соседей.

2. Обновляются все чёрные узлы, используя только что посчитанные значения красных соседей.

Формула для красного полушага ($k \Rightarrow k + \frac{1}{2}$):

$$\tilde\varphi_{i,j}^{red} = \frac{1}{4}(\varphi_{i-1,j}^{k} + \varphi_{i+1,j}^{k} + \varphi_{i,j-1}^{k} + \varphi_{i,j+1}^{k}), (i,j) - Red$$

$$\varphi_{i,j}^{k+\frac{1}{2}} = \varphi_{i,j}^k(1 - \omega) + \tilde\varphi_{i,j}^{red}\omega$$

Формула для чёрного полушага ($k + \frac{1}{2} \Rightarrow k + 1$):

$$\tilde\varphi_{i,j}^{black} = \frac{1}{4}(\varphi_{i-1,j}^{k} + \varphi_{i+1,j}^{k} + \varphi_{i,j-1}^{k} + \varphi_{i,j+1}^{k}), (i,j) - Black$$

$$\varphi_{i,j}^{k+\frac{1}{2}} = \varphi_{i,j}^k(1 - \omega) + \tilde\varphi_{i,j}^{black}\omega$$

После чего получаем приближение, повторяя которое много раз можно добиться качественной апроксимации

\end{document}
